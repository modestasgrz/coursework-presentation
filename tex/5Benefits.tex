\section{Benefits}

 \begin{frame}[c]{Išvados}
    \begin{itemize}
        \item Pagal dabartinį projektavimą CLIP modelis nesugeba išmokti užtikrintai klasifikuoti grafeno sintezės vaizdų pagal oksidatoriaus klases. Visiems klasės spėjimo variantams modelis priskiria apylygias tikimybes.
        \item Difuzijos modelis (generatorius) generuoja panašius vaizdus į tuos, kokių būtų tikimąsi, tačiau šie vaizdai nėra pakankamai detalūs - detaliai aiškūs.
        \item Remiantis paveikslėlių generavimo iš tekstinės įvesties užduotimis (\textit{text2image}), šiuos abu modelius galima adaptuoti grafeno sintezės reakcijų rezultatų prognozavimui.
    \end{itemize}
\end{frame}
